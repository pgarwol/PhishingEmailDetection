\documentclass[]{article}
\usepackage[T1]{fontenc}
\usepackage[utf8]{inputenc}
\usepackage[polish]{babel}
\usepackage[top=1in, bottom=1in, left=1in, right=1in]{geometry}

%opening
\title{Phishing}
\author{Nikodem Kaczmarek, Patryk Garwol}


\begin{document}
\maketitle

\newpage 
\section{Wprowadzenie}

\newpage
\section{Mechanizm działania phishingu}
Phishing to coś tam \cite{whatIsPhishingCNBC}
\newpage
\section{Narzędzia i techniki wykorzystywane przez oszustów}

\newpage
\section{Cele ataków phishingowych}

\newpage
\section{Sposoby identyfikacji phishingu}

\newpage
\section{Skutki ataków phishingowych}

\newpage
\section{Ochrona przed phishingiem}

\newpage
\section{Studium przypadku: Znane ataki phishingowe}

\newpage
\section{Przyszłość phishingu}

\newpage
\section{Podsumowanie}

\newpage
\bibliographystyle{plain}
\bibliography{bibliografia.bib}
\end{document}
